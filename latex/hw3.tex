\documentclass[11pt]{article}
\usepackage{fullpage}
\usepackage{graphicx}
\usepackage{hyperref}
\begin{document}

\title{Homework 3 -- Precomputed Radiance Transfer}
\author{Gabe Fierro and Graham Tremper}
\date{\today}
\maketitle

\section{Introduction}

\section{Basic Relighting}
For the basic relighting viewer, we start by computing the light transport matrix using cubemap pixel lights. A separate transport matrix is used for each color channel. We raytrace the columns of the matrix using the open source POVray ray-tracer with each column representing the scene lit from a single cubemap pixel. This allows us to re-create an arbitrarily lit scene by linearly combining the rows of the transport matrix with the weight of the light at the corresponding cubemap pixel. We can do this in real-time without compression if a limited number of cubemap lights are used.

\section{Environment Maps}
Once we light transport matrix is computing, adding an environment map is straightforward. We used environments from ``Paul Debevec's'' website, including Grace Cathedral, a eucalyptus grove, and a beach scene. Rendering the scene simply involves weighting each column of the transport matrix by the intensity of the corresponding pixel on the cubemap. Without compression, real-time frame rate become intractable with large environment maps and output image resolutions.

\section{Wavelet Transform}
In order to allow for larger scenes with higher resolution environment maps, we used a Haar wavelet transform along each row of the light transport matrix. Since each of these rows is essentially an environment map, we did a 2d Haar transform for each face of the ``cube''. When rendering a frame, we haar transform the current orientation of environment and then perform a sparse matrix multiplication, using only the most significant terms. Ng et al. 03 suggested three methods of selecting subset of wavelet basis lights. These are a nieve ``unweighted selection'' which only looks at the wavelet coefficients, a``transport-weighted selection'' which wights each wavelet coefficient by the energy in the $i^{th}$ column of the transport matrix, and an ``area-weighted selection'' which scales the priority by the area of the wavelet light. The results for these approaches can be seen in the figures below.



\end{document}
